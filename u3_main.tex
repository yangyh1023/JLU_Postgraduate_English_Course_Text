\documentclass[12pt]{article}
\usepackage[utf8]{inputenc}
\usepackage{CJKutf8}
\usepackage{setspace}
\setstretch{1.3}
\usepackage[a4paper,margin=1in]{geometry}

\begin{document}

\begin{CJK}{UTF8}{gbsn}

\section*{UNIT 3}

\textbf{1.} Most Americans would have a difficult time telling you, specifically, what the values are that Americans live by. They have never given the matter much thought.\\

\textbf{1.} 大多数美国人在谈起其赖以生存的价值观时会感到力不从心。他们从未仔细考虑过价值观这个问题。\\

\textbf{2.} Even if Americans had considered this question, they would probably, in the end, decide not to answer in terms of a definitive list of values. The reason for this decision is itself one very American value — their belief that every individual is so unique that the same list of values could never be applied to all, or even most, of their fellow citizens.\\

\textbf{2.} 即使美国人考虑过这个问题,他们最终也不可能决定以一张明确的价值观清单来回答。做出这样的一个决定,本身就是一个非常美国式的价值观——他们相信每个个体都是独一无二的,相同的价值观永远也不可能适用于所有的美国公民,甚至不能适用于大多数公民。\\

\textbf{3.} Although Americans may think of themselves as being more varied and unpredictable than they actually are, it is significant that they think they are. Americans tend to think they have been only slightly influenced by family, church or schools. In the end, each believes, ``I personally chose which values I want to live my own life by.''\\

\textbf{3.} 尽管美国人可能认为他们自己比实际看上去更加变幻莫测,但重要的是他们的确认为自己变幻莫测。美国人普遍认为他们受家庭、教会或学校影响很轻微。最终,每个人都认为“我个人会根据自己生活方式选择我的价值观”。\\

\textbf{4.} The different behaviors of a people or a culture make sense only when seen through the basic beliefs, assumptions and values of that particular group. When you encounter an action, or hear a statement in the United States that surprises you, try to see it as an expression of one or more of the values listed here.\\

\textbf{4.} 一个民族的不同行为方式或者一种文化之所以有意义,是因为人们通过该民族的基本信仰、看法和价值观念来看待它们。在美国,如果某一个行为或某一句话使你感到吃惊,那么你可以将其与下面罗列的价值观对号入座。\\

\textbf{5.} Before proceeding to the list itself, we should also point out that Americans see all of these values as very positive ones. They are not aware, for example, that the people in many Third World countries view some of these values as negative or threatening.In fact, all of these American values are judged by many of the world's citizens as negative and undesirable. Therefore, it is not enough simply to familiarize yourself with these values. You must also, so far as possible, consider them without the negative or derogatory connotation that they might have for you, based on your own experience and cultural identity.\\

\textbf{5.} 在探讨这个清单之前,有必要指出美国人认为这些价值观是充满正能量的。他们没有意识到许多第三世界国家的人们可能认为其中一些价值观是消极或者可怕的。事实上,许多外国人认为美国人的这些价值观是消极和不受欢迎的。因此,仅仅熟悉这些价值观是不够的,还必须尽可能做到不因自身经历和文化身份而对这些价值观有负面和贬损的看法。\\

\subsection*{Personal Control over the Environment\\对环境的自我把握}

\textbf{6.} Americans no longer believe in the power of Fate, and they have come to look at people who do as being backward, primitive, or hopelessly naive. To be called ``fatalistic'' is one of the worst criticisms one can receive in the American context; to an American, it means one is superstitious and lazy, unwilling to take any initiative in bringing about improvement.\\

\textbf{6.} 美国人不再相信命运的力量,那些相信此道的人被认为是落后、原始和极其幼稚的。在美国语境下,“宿命论者”是对一个人最糟糕的评价之一;对美国人来说,这一评价意味着这个人迷信、懒惰且不思进取。\\

\textbf{7.} In the United States, people consider it normal and right that Man should control Nature, rather than the other way around. More specifically, people believe every single individual should have control over whatever in the environment might potentially affect him or her.The problems of one's life are not seen as having resulted from bad luck as much as having come from one's laziness in pursuing a better life. Furthermore, it is considered normal that anyone should look out for his or her own self-interests first and foremost.\\

\textbf{7.} 在美国,人们认为人定胜天,而非受制于自然的观点既正常又正确。更确切地说,人们相信每个人都应该控制周围环境中任何可能影响到自己的因素。一个人在追求美好生活的过程中出现挫折人们不认为是因为运气差,而是由自己的懒惰所导致。而且,人们认为如果一个人把追寻个人利益放在第一位是很正常的。\\

\subsection*{Time and Its Control\\把握时间}

\textbf{8.} Time is, for the average American, of utmost importance. To the foreign visitor, Americans seem to be more concerned with getting things accomplished on time (according to a predetermined schedule) than they are with developing deep interpersonal relations. Schedules, for the American, are meant to be planned and then followed in the smallest detail.\\

\textbf{8.} 对普通美国人来说,时间至关重要。在外国人看来,美国人似乎更关注按时(根据预定的日程表)完成任务而不是发展深层人际关系。美国人认为,哪怕最小的细节也必须在日程安排中列出来并付诸实施。\\

\textbf{9.} It may seem to you that most Americans are completely controlled by the little machines they wear on their wrists, cutting their discussions off abruptly to make it to their next appointment on time.\\

\textbf{9.} 你或许会觉得美国人完全被戴在手腕上的那个小玩意所控制着,为了能准时赴下一个约会,他们会突然打断谈话。\\

\textbf{10.} Americans' language is filled with references to time, giving a clear indication of how much it is valued. Time is something to be ``on,'' to be ``kept,'' ``filled,'' ``saved,'' ``used,'' ``spent,'' ``wasted,'' ``lost,'' ``gained,'' ``planned,'' ``given,'' ``made the most of,'' even ``killed.''\\

\textbf{10.} 美国人的语言中充斥着时间的指示词,这也暗示出人们对于时间的重视程度。时间可以遵守、填补、节省、利用、度过,浪费、失去、赢得、计划、给予,充分利用,甚至可以消磨。\\

\textbf{11.} The international visitor soon learns that it is considered very rude to be late — even by 10 minutes — for an appointment in the United States. (Whenever it is absolutely impossible to be on time, you should phone ahead and tell the person you have been unavoidably detained and will be a half hour — or whatever — late.)\\

\textbf{11.} 外国人很快就会发现,在美国与人约会,即使迟到十分钟就会被认为是很不礼貌的行为。如果实在无法准时到达,应事先打个电话告诉不得不久等你的人你将迟到半小时或怎样。\\

\subsection*{Equality\\平等}

\textbf{12.} Equality is, for Americans, one of their most cherished values. This concept is so important for Americans that they have even given it a religious basis. They say all people have been ``created equal.'' Most Americans believe that God views all humans alike without regard to intelligence, physical condition or economic status.In secular terms this belief is translated into the assertion that all people have an equal opportunity to succeed in life. Americans differ in opinion about how to make this ideal into a reality. Yet virtually all agree that equality is an important civic and social goal.\\

\textbf{12.} 平等是美国人最珍视的价值理念之一,美国人也因此而赋予这一理念以宗教基础。他们说人“生来平等”。大多数美国人相信上帝会平等地看待每一个人,而不考虑其智力、体力或经济方面的差异。通俗地说,这种信仰被解释为一种信念,即每个人都有平等获得成功的机会。美国人在如何把这种理想转化为现实的看法方面存在分歧,然而人们一致认为平等是公民和社会共同追求的重要目标。\\

\textbf{13.} The equality concept often makes Americans seem strange to foreign visitors. Seven-eighths of the world feels quite differently. To them, rank and status and authority are seen as much more desirable considerations — even if they personally happen to find themselves near the bottom of the social order.Class and authority seem to give people in those other societies a sense of security and certainty. People outside the United States consider it reassuring to know, from birth, who they are and where they fit into the complex system called ``society.''\\

\textbf{13.} 这种关于平等的理念常使外国人感到美国人不可思议。十之八九的外国人并不认同这一理念。对他们来说,等级、地位和权威是人们更加向往的东西,即使他们碰巧发现自己接近社会次序的底层。等级和权威似乎能给那些其他国家的人一种安全感和确定感。美国以外的人认为一个人从刚一出生就知道自己是谁、属于被称之为“社会”的复杂体系的哪个阶层会让人感到宽慰。\\

\textbf{14.} Many highly-placed foreign visitors to the United States are insulted by the way they are treated by service personnel (such as waiters in restaurants, clerks in stores, taxi drivers, etc.). Americans have an aversion to treating people of high position in a deferential manner, and, conversely often treat lower class people as if they were very important. Newcomers to the United States should realize that no insult or personal indignity is intended by this lack of deference to rank or position in society. A foreigner should be prepared to be considered ``just like anybody else'' while in the country.\\

\textbf{14.} 很多到美国来的外国权贵会被美国服务人员(例如餐厅里的侍者、商店里的店员和出租车司机)的服务方式所侮辱。美国人不喜欢以一种恭敬的态度对待有权势的人;相反,那些身份不高的人常常会感到自己在美国很受尊重。初到美国的人应该意识到这种对社会等级或地位的漠然中并不包含侮辱和对个人的轻蔑。在美国的外国人要做好被当做“普通人”看待的心理准备。\\

\subsection*{Individualism and Privacy\\个人主义与隐私}

\textbf{15.} Americans think they are more individualist in their thoughts and actions than, in fact, they are. They resist being thought of as representatives of a homogenous group, whatever the group.They may, and do, join groups — in fact many groups — but somehow believe they're just a little different, just a little unique, just a little special, from other members of the same group. And they tend to leave groups as easily as they enter them.\\

\textbf{15.} 美国人认为自己在思想和行动上都是高度个人主义的。他们抗拒担任任何同质组织的典型代表。他们的确会加入一个群体(亦或许多群体),但他们相信他们与众不同、独一无二,与其他团体成员总有区别。而且他们可以随时加入或退出某个团体。\\

\textbf{16.} Privacy, the ultimate result of individualism is perhaps even more difficult for the foreigner to comprehend. The word ``privacy'' does not even exist in many languages. If it does, it is likely to have a strongly negative connotation, suggesting loneliness or isolation from the group. In the United States, privacy is not only seen as a very positive condition, but it is also viewed as a requirement that all humans would find equally necessary, desirable and satisfying. It is not uncommon for Americans to say — and believe — such statements as ``If I don't have at least half an hour a day to myself, I will go stark mad.''\\

\textbf{16.} 外国人也许更难理解个人主义的终极产物——隐私。在许多语言中甚至不存在“隐私”这个字眼。即使有,也可能带有强烈的贬义,暗示孤独寂寞或与其他群体成员格格不入。而在美国,隐私不仅被看做一种非常积极的状态,而且被视为必需品,人们普遍认为其不可或缺,令人向往。难怪美国人会有这样一种说法,而且相信“如果每天我没有至少半个小时留给自己,我将会彻底疯掉”。\\

\subsection*{Action/Work Orientation\\工作至上}

\textbf{17.} ``Don't just stand there,'' goes a typical bit of American advice, ``do something!'' This expression is normally used in a crisis situation, yet, in a sense, it describes most American's entire waking life, where action — any action — is seen to be superior to inaction.\\

\textbf{17.} 一句典型的美国忠告:“不要光说不练,做点什么!”虽然这句话通常用在危机状况中,但在一定意义上说明了美国人积极的生活方式。在美国人的生活中,认为活动——任何活动——都比不作为要好。\\

\textbf{18.} Americans routinely plan and schedule an extremely active day. Any relaxation must be limited in time, pre-planned, and aimed at ``recreating'' their ability to work harder and more productively once the recreation is over.Americans believe leisure activities should consume a relatively small portion of one's total life. People think that it is ``sinful'' to ``waste one's time,'' ``to sit around doing nothing,'' or just to ``daydream.''\\

\textbf{18.} 美国人的一天通常会安排得很紧凑。任何放松消遣都必须限制在一定时间内,而且这也是为了“更好地工作”,因此一旦他们结束“放松”之后他们会更加努力地工作。美国人认为休闲活动应该只占人生的一小部分。人们认为“浪费时间”、“无所事事”或“白日做梦”是“大逆不道的”。\\

\textbf{19.} Such a ``no nonsense'' attitude toward life has created many people who have come to be known as ``workaholics'' or people who are addicted to their work, who think constantly about their jobs and who are frustrated if they are kept away from them, even during their evening hours and weekends.\\

\textbf{19.} 这样一个“不做无意义的事”的人生态度造就了一大批被称作“工作狂”的人——他们沉醉于,甚至常常全身心投入到他们的工作或职业当中。他们一心扑在工作上,一旦离开工作,就会感到身心受挫。因此,晚上下班后或周末不上班他们会感到不自在。\\

\textbf{20.} The workaholic syndrome, in turn, causes Americans to identify themselves wholly with their professions. The first question one American will ask another American when meeting for the first time is related to his or her work: ``Where do you work?'' or ``Who (what company) are you with?''\\

\textbf{20.} 相应地,这种对工作的狂热使美国人完全把自己和职业划上等号。在美国,见面问到的第一个问题一般都与工作有关:``在哪里发财?''或者``在哪个单位上班?''\\

\textbf{21.} And when such a person finally goes on vacation, even the vacation will be carefully planned, very busy and active.\\

\textbf{21.} 这种人如果去度假,他们会提前做出假期周密计划,使假期过得紧凑而有趣。\\

\textbf{22.} America may be one of the few countries in the world where it seems reasonable to speak about the ``dignity of human labor'', meaning by that, hard, physical labor. In America, even corporation presidents will engage in physical labor from time to time and gain, rather than lose, respect from others for such action.\\

\textbf{22.} 美国可能是世界上仅有的几个人们认为体力劳动光荣的国家。即使是公司老总也会时不时参加体力劳动,而且在这种情况下,会得到别人的尊重而不是鄙视。\\

\subsection*{Informality\\不拘小节}

\textbf{23.} If you come from a more formal society, you will likely find Americans to be extremely informal, and will probably feel that they are even disrespectful of those in authority. Americans are one of the most informal and casual people in the world, even when compared to their near relative — the Western European.\\

\textbf{23.} 如果你来自一个更遵守礼仪的国家,你可能会觉得美国人极端不拘礼节,甚至有点目无尊长。即使与其近邻西欧相比,美国也称得上是世界上最随便的民族之一。\\

\textbf{24.} As one example of this informality, American bosses often urge their employees to call them by their first names and even feel uncomfortable if they are called by the title ``Mr.'' or ``Mrs.''\\

\textbf{24.} 举例说来,美国老板常要求他们的雇员直呼其名。如果以``某某先生''或``某某夫人''相称,他们会感到全身不自在。\\

\textbf{25.} Dress is another area where American informality will be most noticeable, perhaps even shocking. One can go to a symphony performance, for example, in any large American city nowadays and find some people in the audience dressed in blue jeans and tieless, short-sleeved shirts.\\

\textbf{25.} 美国人的这一风格在衣着方面体现得淋漓尽致,甚至令人叹为观止。如果你走进当今美国大都市的交响音乐会,便会发现,观众中有很多人都穿着牛仔裤,着短袖衫且不系领带。\\

\textbf{26.} Informality is also apparent in American's greetings. The more formal ``How are you?'' has largely been replaced with an informal ``Hi.'' This is as likely to be used to one's superior as to one's best friend.\\

\textbf{26.} 这种不拘小节的做法在同人打招呼时也可见一斑:较正式的问候``你好''已被更为随意的``嘿''所代替,不管对方是自己的顶头上司还是亲密好友。\\

\textbf{27.} If you are a highly placed official in your own country, you will probably, at first, find such informality to be very unsettling. American, on the other hand, would consider such informality as a compliment! Certainly it is not intended as an insult and should not be taken as such.\\

\textbf{27.} 如果你是一个位高权重之人,一开始你可能无法容忍这种不拘小节。而美国人却将其看作是一种恭维与赞誉。当然这种不拘小节绝非有意冒犯。\\

\subsection*{Directness, Openness and Honesty\\直接、率真、诚实}

\textbf{28.} Many other countries have developed subtle, sometimes highly ritualistic, ways of informing other people of unpleasant information. Americans, however, have always preferred the first approach.They are likely to be completely honest in delivering their negative evaluations. If you come from a society that uses the indirect manner of conveying bad news or uncomplimentary evaluations, you will be shocked at Americans' bluntness.\\

\textbf{28.} 很多美国以外的国家都积累了委婉的、有时是高度程式化的语言来传达不好的消息,可美国人永远最爱最直接的方式。他们在表达自己的负面看法时绝对诚实。如果在你的国家人们用间接委婉的方式来传达坏消息或负面评价,美国人的直率简直会令习惯隐喻的你大惊失色!\\

\textbf{29.} If you come from a country where saving face is important, be assured that Americans are not trying to make you lose face with their directness. It is important to realize that an American would not, in such case, lose face.The burden of adjustment, in all cases while you are in this country, will be on you. There is no way to soften the blow of such directness and openness if you are not used to it except to tell you that the rules have changed while you are here.Indeed, Americans are trying to urge their fellow countrymen to become even more open and direct. The large number of ``assertiveness'' training courses that appeared in the United States in the late 1970s reflects such a commitment.\\

\textbf{29.} 如果你认为“要面子”很重要,那么请相信美国人的直率并不是存心让你大丢面子。而且你要明白美国人也不会因此而丢面子。不管怎样,一旦你来到美国,学会入乡随俗是你必备功课。如果你不习惯,对不起,没有办法缓和这种直接和率真的风气,只能说,你来到这里,习惯已经改了。美国人正试图敦促自己的同胞更加开放和直率。从70年代末期美国大量涌现的“自信培训班”是这一趋势的鲜明写照。\\

\textbf{30.} Americans consider anything other than the most direct and open approach to be dishonest and insincere and will quickly lose confidence in and distrust anyone who hints at what is intended rather than saying it outright.\\

\textbf{30.} 美国人认为如果说话做事不直截了当就是缺乏诚意,他们不会信任那些有话不直说而专爱拐弯抹角之人。\\

\textbf{31.} Anyone who, in the United States, chooses to use an intermediary to deliver that message will also be considered manipulative and untrustworthy.\\

\textbf{31.} 在美国,人们通常给那些说话拐弯抹角的人扣上一顶“世故圆滑、不可信任”的帽子。\\

\subsection*{Practicality and Efficiency\\务实和效率}

\textbf{32.} Americans have a reputation for being realistic, practical, and efficient. The practical consideration is likely to be given highest priority in making any important decision.Americans pride themselves in not being very philosophically or theoretically oriented. If Americans would even admit to having a philosophy, it would probably be that of pragmatism.Will it make money? What is the bottom line? What can I gain from this activity? These are the kinds of questions Americans are likely to ask, rather than: is it aesthetically pleasing? Will it be enjoyable? Will it advance the cause of knowledge?This pragmatic orientation has caused Americans to contribute more inventions to the world than any other country in human history. The love of ``practicality'' has also caused Americans to view some professions more favorably than others.Management and economics are much more popular in the United States than philosophy or anthropology, and law and medicine more valued than the arts.Americans belittle ``emotional'' and ``subjective'' evaluations in favor of ``rational'' and ``objective'' assessments. Americans try to avoid being ``too sentimental'' in making their decisions. They judge every situation ``on its own merits.''\\

\textbf{32.} 美国人被公认是务实和讲效率的。人们在做出重大决定时,往往首先便会考虑这样做是否行之有效。美国人为自己不是很哲学化或理论化而感到自豪。如果说美国人也会承认他们尊崇一派哲学,那便只可能是实用主义哲学了。这样做能挣钱吗?最坏会是什么结果?我能从中得到什么?这些都是美国人在决策之前最常问的问题,而不是诸如:这样做体面吗?有趣吗?能推进知识的发展吗?之类的问题。这种务实倾向使得美国人在发明创新上胜过其他世界各国。这种对“实用性”的热爱也导致了美国人偏爱某些职业胜于其他职业。管理和经济学在美国比哲学和人类学更受欢迎,同样法律和医学比艺术更受重视。美国人鄙视主观感性,偏爱理性客观。美国人在做决定时绝对不会感情用事。他们一定会就事论事,因事而异。\\

\subsection*{英译汉}

\textbf{1)} Foreigners generally consider Americans much more materialistic than Americans are likely to consider themselves. Americans would like to think that their material objects are just the ``natural benefits'' that result from hard work and serious intent - a reward, they think, which all people could enjoy were they as industrious and hard-working as Americans. But by any standard, Americans are materialistic.They give a higher priority to obtaining, maintaining, and protecting material objects than they do in developing and enjoying relationships with people. Since Americans value newness and innovation, they sell or throw away their possessions frequently and replace them with newer ones. A car may be kept for only two or three years, a house for five or six before buying a new one.\\

外国人普遍认为美国人比他们自身可能认为的还要注重物质。美国人喜欢认为他们的物质对象就是“天然的福利”——一种对努力工作和真诚目标的回报,所有人都可以享受的到,只要他们像美国人一样勤奋并努力工作。但是,无论以何种标准考量,美国人都是物质至上的。他们对获取、维修和保护物质对象的重视程度高于培养和享受人际关系。因为美国人重视新奇和创新,所以他们频繁扔掉一些东西代之以新的物件。比如一辆车可能只开两三年,一栋房子只住五六年就要换新的。\\

\subsection*{汉译英}

\textbf{1)} 文化交流是个互相学习、互相借鉴的过程。中国文化既要扎根本土、坚持传统,又要紧跟时代。我们心悦诚服地赞赏其他国家、民族的先进文化,并以博大的胸怀博采众长。这是发展我们民族的、大众的、社会主义文化的需要。数百年来,中国文化令西方人感到神秘,又使他们心驰神往。今天,在“古老的中国、多彩的中国、现代的中国”这一主题统领下,中国传统文化、民族文化和社会文化、现代文化将使世界人民得到一次更全面、更深刻的精神享受,它向世界公众展示的是中国文化异彩纷呈、百花齐放的繁荣景象。\\

The cultural exchange is a process of learning and drawing on experience from each other. The Chinese culture must keep its locality and tradition on the one hand, and step up with the times on the other. We admire from the heart the advanced cultures of other countries and nations and would like to learn from them openheartedly. This just meets the needs of developing our national, mass and socialist culture.For hundreds of years, people in the West not only marveled at the mystery of Chinese culture but also have a deep longing for it. Now, under the theme ``Ancient China, Colorful China, and Modern China'', it will offer the world's people a spiritual enjoyment of the Chinese traditional culture, ethic and local culture, and modern culture in an all-round and deeper way. It has displayed to the world colorful and varied culture of China, a flourishing vista in which hundreds of flowers are vying with one another to blossom.\\

\end{CJK}
\end{document}