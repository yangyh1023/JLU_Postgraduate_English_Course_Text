\documentclass[12pt]{article}
\usepackage[utf8]{inputenc}
\usepackage{CJKutf8}
\usepackage{setspace}
\setstretch{1.3}
\usepackage[a4paper,margin=1in]{geometry}

\begin{document}

\begin{CJK}{UTF8}{gbsn}

\section*{UNIT 4}

\textbf{1.} Think for a moment about your own life — the activities of your day, the possessions you enjoy, the surroundings in which you live. Is there anything you don't have at this moment that you would like to have? Anything that you have, but that you would like more of? If your answer is ``no,'' then congratulations — either you are well advanced on the path of Zen self-denial, or else you are a close relative of Ted Turner. The rest of us, however, would benefit from an increase in our material standard of living. This simple truth is at the very core of economics. It can be restated this way: we all face the problem of scarcity.\\

\textbf{1.} 想一想你的生活:你每天从事的活动,你所拥有的财产,你所居住的环境。此时此刻,你是否希望拥有一些你所没有的东西?对于那些你已经拥有的,你是否希望拥有更多?如果你的答案是否定的,那么恭喜你:你要么早已看破红尘,要么就是腰缠万贯。然而,我们大多数人的答案却是肯定的,我们都希望拥有更多,从而进一步提高物质生活水平。这一简单的真理就是经济学的核心。我们或许可以重新表述这个问题:我们都面临稀缺。\\

\textbf{2.} Almost everything in your daily life is scarce. You would benefit from a larger room or apartment, so you have a scarcity of space. You have only two pairs of shoes and could use a third for hiking; you have a scarcity of shoes. You would love to take a trip to Chicago, but it is difficult for you to find the time or the money to go — trips to Chicago are scarce.\\

\textbf{2.} 几乎日常生活中的一切都是稀缺不足的。比如说你希望你的房间或公寓能再大点儿,那么对你而言,居住空间就是稀缺的;比如说你只有两双鞋,你还想拥有一双适于徒步旅行,那么鞋子对你来说就是稀缺的;再比如说你很想去趟芝加哥,可是你既没有余钱也没有空闲去,那么芝加哥之行对你而言就是稀缺的。\\

\textbf{3.} Because of scarcity, each of us is forced to make choices. We must allocate our scarce time to different activities: work, play, education, sleep, shopping, and more. We must allocate our scarce spending power among different goods and services: food, furniture, movies, long-distance phone calls, and many others.\\

\textbf{3.} 因为稀缺,我们不得不进行选择。我们不得不把稀缺的时间在工作、娱乐休闲、教育、睡眠、购物等不同的活动中进行分配;我们也不得不把稀缺的金钱在食物、家具、电影、长途电话等不同的产品和劳务中进行分配。\\

\textbf{4.} Economists study the choices we make as individuals and how those choices shape our economy. For example, the goods that each of us decides to buy ultimately determine which goods business firms will produce. This, in turn, explains which firms and industries will hire new workers and which will lay them off.\\

\textbf{4.} 经济学家研究社会中个体的选择及这些选择对个体经济状况的影响。例如,个体对产品的选择最终决定了企业生产什么产品,这继而又决定了哪些企业要招工,哪些企业要裁员。\\

\textbf{5.} Economists also study the more subtle and indirect effects of individual choice on our society. Will most Americans continue to live in houses, or like Europeans will most of us end up in apartments? Will we have an educated and well-informed citizenry? Will museums and libraries be forced to close down? Will traffic congestion in our cities continue to worsen, or is there relief in sight? These questions hinge, in large part, on the separate decisions of millions of people. To answer them requires an understanding of how people make choices under conditions of scarcity.\\

\textbf{5.} 经济学家也研究个体的选择对社会所产生的细微的间接的影响。多数美国人的家的归宿是独立的房子还是像欧洲人那样的公寓?国民受教育程度会越来越高吗?博物馆和图书馆以后都得关门歇业吗?城市里的交通堵塞会越来越严重还是会缓解在即呢?这些问题在很大程度上都取决于成千上万个个体的决定。若想回答这些问题必须明白人们如何在稀缺的前提下进行选择。\\

\textbf{6.} Think for a moment about the goals of our society. We want a high standard of living for all citizens: clean air, safe streets, and good schools. What is holding us back from accomplishing all of these goals in a way that would satisfy everyone? You probably already know the answer: scarcity.\\

\textbf{6.} 想一想我们社会的目标:全体国民生活水平的提高、清新的空气、良好的治安、好的学校。是什么原因使我们不能实现这所有的目标而使人人都满意呢?你想必已经知道答案了:稀缺。\\

\textbf{7.} Society's problem is a scarcity of resources — the things we use to make goods and services. Economists classify resources into three categories: labor, capital, land. Anything produced in the economy comes, ultimately, from some combination of these resources. Think about the last lecture you attended at your college. You were consuming a service — a college lecture. What went into producing that service? Labor was supplied by your instructor. Many types of capital were used as well. The physical capital included desks, chairs, a blackboard or transparency projector, and the classroom building itself. It also included the computer your instructor may have used to write out his or her lecture notes.In addition, there was human capital — your instructor's specialized knowledge and lecturing skills. Finally, there was land — the property on which your classroom building sits. These very same resources, however, could instead be used to produce other desirable things, such as primary schools, hospitals. As a result, every society must have some method of allocating its scarce resources — choosing which of our many competing desires will be fulfilled and which will not be.\\

\textbf{7.} 社会的稀缺主要是资源的稀缺。资源是指我们用于生产产品和劳务的东西。经济学家把资源分为三类:劳动力、资本和土地。经济社会中任何产品和劳务的生产都需要这三种资源。比如你刚听过的一节大学课,你正在消费一种劳务——大学讲课。这种劳务都需要什么资源呢?既需要教师提供劳动力,也需要各种各样的资本,如桌子、椅子、黑板、电脑、投影仪、教室等物质资本。它还需要人力资本,即教师的专业知识和讲课技巧。最后,还需要教室所占用的土地。然而,这些资源亦可用于生产其它人们想要的东西,如新的民宅、医院、汽车厂,或是学校。因此,每一个社会都必须采用一些分配其稀缺资源的方法一选择满足哪些最紧迫的需求。\\

\textbf{8.} Many of the big questions of our time center on the different ways in which resources can be allocated. The cataclysmic changes taking place in Eastern Europe and the former Soviet Union arose from a very simple fact: the method these countries used to allocate resources was not working.\\

\textbf{8.} 我们这个时代的许多大问题都与资源的分配密切相关。东欧和苏联巨变的根本原因只有一个:这些国家资源分配方式行之无效。\\

\textbf{9.} What does it cost you to go to the movies? If you answered seven or eight dollars, because that is the price of a movie ticket, then you are leaving a lot out. Most of us are used to thinking of cost as the money we must pay for something. A Big Mac costs 2.50,anewToyotaCorollacosts15,000, and so on. Certainly, the money we pay for goods or service is a part of its cost, but it is not necessarily the entire cost or even the largest part of the cost. Economics takes a broader view of costs, recognizing monetary as well as nonmonetary components.\\

\textbf{9.} 看一场电影的成本是多少?如果只因电影票的价钱是7或8美元,你就回答说7或8美元,那么你就大错特错了。的确,我们已经习惯了把支付了多少钱视为购买某物的成本,如买一个巨无霸要付2.5美元,买一辆新的丰田花冠要付1万5千美元。可是我们为一件产品或一项劳务所支付的钱并不是其全部成本,有的甚至只占其全部成本的极小份额。经济学家赋予成本更广泛的含义,涵概了钱和非钱的因素。\\

\textbf{10.} The total cost of any action — buying a car, producing a computer, or even reading a book — is what we must give up when we take that action. This cost is called the opportunity cost of the action, because any economic activity uses up scarce resources and therefore requires us to give up the opportunity to enjoy other things for which those resources could have been used.\\

\textbf{10.} 任何行为的全部成本一购买汽车、电脑,甚至是读书一都是我们为此所必须放弃的一切。因为任何行为都要耗掉稀缺的资源而迫使我们放弃享受其它事物(其它事物也需要这些资源)的机会,所以这种对其它机会的放弃被称为该行为的机会成本。\\

\textbf{11.} The opportunity cost of any choice is what we give up when we make that choice.\\

\textbf{11.} 任何选择的机会成本就是为其所放弃的其它机会。\\

\textbf{12.} Opportunity cost is the concept of cost that should be used in decision making.\\

\textbf{12.} 在决策时,人们应考虑的是机会成本。\\

\textbf{13.} Virtually every action we take as individuals uses up scarce money or scarce time or both. Hence, every action we choose requires us to sacrifice other enjoyable goods, services, and activities for which we could have used our time and money.For example, it took a substantial amount of the authors' time to write this textbook. Suppose that the time devoted to writing the book could instead have been used by one of the authors to either (1) go to law school, (2) write a novel, or (3) start a profitable business.\\

\textbf{13.} 几乎个人的每一个行为都要耗掉稀缺的金钱或时间或二者兼而有之,因此我们的每一次选择都迫使我们牺牲掉其它的产品和劳务,因为那些也需要时间和金钱。例如,写这本教材需要作者很多时间,而这些时间亦可以用于(1)攻读法律,(2)写一本小说,(3)经商。\\

\textbf{14.} Do all of these alternatives together make up the opportunity cost of writing this book? Not really. The time released from not writing the book would not be sufficient to pursue all of these activities. Only those alternatives that would actually have been chosen should be identified as the opportunity cost of writing the book. But which one would have been chosen? The one that is next most attractive to the decision maker. The opportunity cost of any choice, then, is the next most attractive alternative that must be sacrificed. The other less valuable alternatives would not have been chosen and therefore play no role in making a decision.\\

\textbf{14.} 所有以上被放弃的选择机会都是写这本教材的成本吗?非也。放弃编写教材而空出来的时间并不足以完成以上三项可能的机会。只有那些真的可能被选中去做的才是编写本教材的机会成本。但是到底哪一项会被选中呢?对决策者第二有吸引力的那一项。所以任何选择的机会成本就是第二有吸引力的,但为了该选择而不得不放弃的选择机会。其它的不够有吸引力的机会就被排除掉,在决策时也不必考虑进成本内。\\

\textbf{15.} To explore this notion of opportunity cost further, let's go back to an earlier question: What does it cost to see a movie? Suppose some friends ask Jessica to a movie located 10 minutes from campus. To see the movie, Jessica will use up scarce funds to buy the movie ticket and scarce time traveling to and from the movie and sitting through it. The money she uses for the movie ticket would otherwise have been spent on a long-distance phone call to a friend in Italy — her next best use of the money — and the time would otherwise have been devoted to studying for her economics exam — her next best use of time. For Jessica, then, the opportunity cost of the movie consists of two things: (1) a phone call to her friend and (2) a higher score on her economics exam. Seeing the movie will cost Jessica to sacrifice both of these valuable alternatives, since the movie will cost Jessica both money and time.\\

\textbf{15.} 为了进一步更好的理解机会成本的概念,让我们回到最初的问题:看一场电影的成本是多少?假设杰西卡的朋友邀请她一起去电影院(距校园路程约为10分钟)看一场电影。为了看电影,杰西卡需要花费稀缺的金钱去买电影票,要花费稀缺的时间去电影院并看电影。她买电影票的钱如果不花的话,就可以给她意大利的一位朋友打个长途电话,此乃该钱的第二最佳用途。她看电影的时间如果不花费的话,她就可以用这些时间来复习功课,为经济学考试做准备,此乃该时间的第二最佳用途。因此,对于杰西卡而言,看一场电影的机会成本由两部分组成:(1)打给朋友的电话;(2)更高的经济学考试成绩。看这场电影需要杰西卡同时放弃二者,因为看电影既要花钱,也要花费时间。\\

\textbf{16.} Now consider Samantha, a highly paid consultant who lives in New York City, several miles from the theater, and who has a backlog of projects to work on. As in Jessica's case, seeing the movie will use scarce funds and scarce time. But for Samantha, the particulars will be different. First, the money costs are greater. There is not only the price of the movie ticket, but also the round-trip cab fare, which could bring the direct money cost to \$20. However, this is only a small part of Samantha's opportunity cost.
Let's suppose that the time it takes Samantha to find out when and where the movie is playing, hail a cab, travel to the movie theater, wait in line, sit through the previews, and travel back home is four hours — not unrealistic for seeing a movie in Manhattan.
Samantha's next best alternative for using her time would be to work on her consulting projects, for which she would earn \$150 per hour. In this case, We can measure the entire opportunity cost of the movie in monetary terms: first, the direct money costs of the movie and cab fare (\$20), and second, the forgone income associated with seeing the movie (\$150$\times$4hours =\$600), for a total of \$620!\\

\textbf{16.} 现在再让我们看一下瑟曼莎看一场电影的成本是多少。与作为学生的杰西卡不同,瑟曼莎是一位住在纽约的有着较高收入的咨询师,她的住所离电影院有几英里远,她的手头还积压着好多项目要做。对于杰西卡而言,看电影要花费她稀缺的时间和金钱,但对于瑟曼莎而言,具体的花费则有所不同。首先,金钱成本更高:不仅需要花钱买电影票,来去还要花钱打车,二者相加为20美元。然而这仅是她看电影的机会成本的一小部分而已。
我们假设她需要花时间寻找具体的播放时间和播放地点,她要叫车并坐车去电影院,她要排队等候,然后要看预告片,然后还要坐车回家,这些加起来至少需要4个小时(这在曼哈顿完全是可能的)。
瑟曼莎这4个小时的第二用途是她的咨询项目工作(报酬是每小时150美元)。在此,我们可以以金钱的方式来衡量一下瑟曼莎看一场电影的全部机会成本:(1)电影票和打车的直接花费是20美元;(2)由于看电影而放弃的收入是150美元×4小时=600美元,二者总计高达620美元。\\

\textbf{17.} At such a high price, you might wonder why Samantha would ever decide to see a movie. Indeed, the same reasoning applies to almost everything Samantha does besides work. It is very expensive for Samantha to talk to a friend on the phone, eat dinner or even sleep — all of these activities require her to sacrifice the direct money costs plus another \$150 per hour of forgone income. Should Samantha ever choose to pursue any of these activities?The answer for Samantha is the same as for Jessica or anyone else: yes — if the activity is more highly valued than what is given up. It is not hard to imagine that, after putting in a long day at work, leisure activities would be very important to Samantha — worth the money cost and the forgone income required to enjoy them.\\

\textbf{17.} 你或许在想为什么瑟曼莎要决定去看一场如此“昂贵”的电影。事实上,此理可推及瑟曼莎工作以外的一切活动:对瑟曼莎而言,和朋友煲电话粥,出席晚宴,甚至睡觉休息都非常“昂贵”,因为所有这些活动都需要她直接花费金钱,而且更为重要的是需要她放弃每小时150美元的收入。既然如此“昂贵”,瑟曼莎是否应该从事这些活动呢?无论对于瑟曼莎,还是杰西卡或其它人答案都是相同的:应该——如果这些活动的价值高于放弃的。不难想象,经过了一整天辛苦的工作,休闲娱乐对瑟曼莎来说是非常重要的——其价值超过机会成本。\\

\textbf{18.} With an understanding of the concept of opportunity cost and how it can differ among different individuals, you can understand some behavior that might otherwise appear strange. For example, why do high-income people rarely shop at discount stores like Kmart, preferring full-service stores where the same items carry much higher price tags? It's not that high-income people like to pay more for their purchases, but that discount stores are generally understaffed and crowded with customers, so shopping there takes more time. While discount stores offer a lower money price, they impose a higher time cost. For high-income people, these stores are actually more costly than stores with higher price tags.We can also understand why the most highly paid consultants, entrepreneurs, attorneys, and surgeons often lead such frenetic lives, doing several things at once and packing every spare minute with tasks. Since these people can earn several hundred dollars for an hour of work, every activity they undertake carries a correspondingly high opportunity cost. Brushing one's teeth can cost \$10, and driving to work can cost hundreds! By combining activities — making phone calls while driving to work, thinking about and planning the day while in the shower, or reading the morning paper in the elevator — the opportunity cost of these routine activities can be minimized.\\

\textbf{18.} 理解了机会成本的概念及不同人从事同一活动的机会成本不同以后,你就能够理解某些看似奇怪的行为。比如说,收入较高的人很少去Kmart这样的大型连锁超市,而偏爱那些全方位服务的商店。这并不是因为这些人愿意多付钱,而是因为一般来讲,在大超市里服务员人数较少(因为是自选),而顾客人数较多(因为便宜),所以在那里购物需要花费更多的时间。虽然超市的商品价格低,但时间成本高,所以对收入较高的人来说,超市里的东西实际上要比商店里的贵。我们也能理解为什么很多收入较高的咨询师、商人、律师、外科医生都是工作狂——抓紧一切时间工作,并总是同时做好几件事。因为这些人每小时的工作都可以赚几百美元,所以他们所从事的其它活动的机会成本相应的也比较高。刷牙的机会成本是10美元,开车上班的机会成本要成百上千美元。如果驾车上班时打电话,冲凉时考虑并计划一天的事情,或是在乘电梯时读早报,这些日常活动的机会成本就可以最小化。\\

\textbf{19.} From an individual's point of view, it is useful to think of opportunity cost as arising from the scarcity of time or money; for society as a whole, it arises from the scarcity of society's resources. Since human wants are unlimited, while society's resources are not, no society can produce enough of everything to satisfy everyone's desires simultaneously. Therefore, all production carries an opportunity cost. To produce and enjoy more of one thing, we must shift resources away from producing something else.\\

\textbf{19.} 对于个人而言,机会成本的产生是由于时间和金钱的稀缺,对于社会整体而言,是由于社会资源的稀缺。人的欲望是无止境的,而资源却是有限的,任何一个社会都不可能同时生产足够的产品和劳务来满足所有人的愿望。因此,所有的生产都孕含着机会成本:若生产和享用一种,必须减少或放弃生产和享用另外一种。\\

\textbf{20.} Consider a goal on which we can all agree: better health for our citizens. What would be needed to achieve this goal? More medical checkups for more people and greater access to top-flight medicine when necessary. These, in turn, would require more and better trained doctors, more hospital buildings and laboratories, and more high-tech medical equipment such as CAT scanners and surgical lasers.In order for us to produce these goods and services, we would have to pull resources — land, labor, machinery, and raw materials — out of producing other goods and services that we also enjoy. The opportunity cost of improved health care, then, consists of these other goods and services we would have to do without.\\

\textbf{20.} 我们都有一个共同目标:为全体公民提供更好的医疗保健。如何实现该目标?这就需要为更多的人提供更多次数和更全面的医疗检查,更多的在必要时使用好药的机会,这继而需要更多的和更好的医生,更多的医院和实验室,更多的高科技的医疗设备(如CAT扫描仪和外科激光手术)。为了生产这些产品和劳务,我们就不得不放弃土地、劳动力、机械、原材料等资源用于生产其它我们也想要的产品和劳务的机会。提高医疗保健的机会成本就是我们所放弃的其它的产品和劳务。\\

\textbf{21.} Opportunity cost is one of the most important ideas you will encounter in economics. The concept sheds light on virtually every problem that economists consider, whether it be explaining the behavior of consumers or business firms or understanding important social problems like poverty or racial discrimination. In all of these cases, economists apply the principle of opportunity cost.\\

\textbf{21.} 机会成本是经济学中的一个重要概念。这一概念有助于理解一切经济问题,包括从消费者或公司的行为选择到贫困或种族歧视等一些重大社会问题。在解释以上这些问题时,经济学家都采用了机会成本的原则。\\

\textbf{22.} The Principle of Opportunity Cost: All economic decisions made by individuals or society are costly. The correct way to measure the cost of a choice is its opportunity cost — that which is given up to make the choice.\\

\textbf{22.} 机会成本原则:无论是个人还是社会所做的经济决策都是有成本的。衡量其成本的正确方式是衡量该决策的机会成本——为选择该项所放弃的其它选择机会。\\

英译汉\\
\textbf{1)} If your answer is ``no,'' then congratulations—either you are well advanced on the path of Zen self-denial, or else you are a close relative of Ted Turner. The rest of us, however, would benefit from an increase in our material standard of living. This simple truth is at the very core of economics. It can be restated this way: we all face the problem of scarcity.\\

如果你的答案是否定的,那么恭喜你:你要么早已看破红尘,要么就是腰缠万贯。然而,我们大多数人的答案却是肯定的,我们都希望拥有更多,从而进一步提高物质生活水平。这一简单的真理就是经济学的核心。我们或许可以重新表述这个问题:我们都面临稀缺。\\

\textbf{2) } Economists also study the more subtle and indirect effects of individual choice on our society. Will most Americans continue to live in houses, or—like Europeans will most of us end up in apartments? Will we have an educated and well-informed citizenry? Will museums and libraries be forced to close down?
Will traffic congestion in our cities continue to worsen, or is there relief in sight? These questions hinge, in large part, on the separate decisions of millions of people. To answer them requires an understanding of how people make choices under conditions of scarcity.\\

经济学家也研究个体的选择对社会所产生的细微的间接的影响。多数美国人的家的归宿是独立的房子还是像欧洲人那样的公寓?国民受教育程度会越来越高吗?博物馆和图书馆以后都得关门歇业吗?城市里的交通堵塞会越来越严重还是会缓解在即呢?这些问题在很大程度上都取决于成千上万个个体的决定。若想回答这些问题必须明白人们如何在稀缺的前提下进行选择。\\

\textbf{3) } Opportunity cost is one of the most important ideas you will encounter in economics. The concept sheds light on virtually every problem that economists consider, whether it be explaining the behavior of consumers or business firms or understanding important social problems like poverty or racial discrimination. In all of these cases, economists apply the principle of opportunity cost.\\

机会成本是经济学中的一个重要概念。这一概念有助于理解一切经济问题,包括从消费者或公司的行为选择到贫困或种族歧视等一些重大社会问题。在解释以上这些问题时,经济学家都采用了机会成本的原则。\\

\end{CJK}
\end{document}
