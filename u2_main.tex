\documentclass[12pt]{article}
\usepackage[utf8]{inputenc}
\usepackage{CJKutf8}
\usepackage{setspace}
\setstretch{1.3}
\usepackage[a4paper,margin=1in]{geometry}

\begin{document}

\begin{CJK}{UTF8}{gbsn}

\section*{UNIT 2}

\textbf{1.} The most important day I remember in all my life is the one on which my teacher, Anne Mansfield Sullivan, came to me. I am filled with wonder when I consider the immeasurable contrasts between the two lives which it connects. It was the third of March, 1887, three months before I was seven years old. \\

\textbf{1.} 在我的一生中,最重要的一天就是我的老师——安妮·莎莉文来到我家的那一天。当我想到这一天连接的两种截然不同的生活时,心中充满了惊奇。那天是1887年3月3日,再过三个月我就满七岁了。 \\

\textbf{2.} The morning after my teacher came she led me into her room and gave me a doll. The little blind children at the Perkins Institution had sent it and Laura Bridgman had dressed it; but I did not know this until afterward. When I had played with it a little while, Miss Sullivan slowly spelled into my hand the word ``d-o-l-l.'' I was at once interested in this finger play and tried to imitate it. When I finally succeeded in making the letters correctly I was flushed with childish pleasure and pride. Running downstairs to my mother I held up my hand and made the letters for doll. I did not know that I was spelling a word or even that words existed; I was simply making my fingers go in monkey-like imitation. In the days that followed I learned to spell in this uncomprehending way a great many words, among them pin, hat, cup and a few verbs like sit, stand and walk. But my teacher had been with me several weeks before I understood that everything has a name. \\

\textbf{2.} 次日早晨,我的老师领我来到她的房间,给了我一个布娃娃。后来我才知道,这个娃娃是柏金斯盲人学院的小盲童们送的,劳拉·布里奇曼为它缝制了衣服。我玩了一会儿娃娃后,莎莉文小姐慢慢地在我手上拼写“doll(玩偶)”这个词。我立刻对这种手指游戏产生了兴趣,并努力模仿她。当我最终成功拼对字母时,心中充满了孩童般的喜悦与自豪。我跑到楼下,举起手在母亲面前拼出了“doll(玩偶)”的字母。当时我并不知道自己在拼写单词,甚至不知道文字的存在,只是像猴子一样模仿着动手指。在随后的几天里,我用这种不求甚解的方式学会了拼写许多单词,比如“pin(针)”“hat(帽子)”“cup(杯子)”,还有“sit(坐)”“stand(站立)”“walk(走)”等几个动词。但直到老师和我相处了几个星期后,我才明白万物都有自己的名字。 \\

\textbf{3.} One day, while I was playing with my new doll, Miss Sullivan put my big rag doll into my lap also, spelled ``d-o-l-l'' and tried to make me understand that ``d-o-l-l'' applied to both. Earlier in the day we had had a tussle over the words ``m-u-g'' and ``w-a-t-e-r.'' Miss Sullivan had tried to impress it upon me that ``m-u-g'' is mug and that ``w-a-t-e-r'' is water, but I persisted in confounding the two. In despair she had dropped the subject for the time, only to renew it at the first opportunity. I became impatient at her repeated attempts and, seizing the new doll, I dashed it upon the floor. I was keenly delighted when I felt the fragments of the broken doll at my feet. Neither sorrow nor regret followed my passionate outburst. I had not loved the doll. In the still, dark world in which I lived there was no strong sentiment of tenderness. I felt my teacher sweep the fragments to one side of the hearth, and I had a sense of satisfaction that the cause of my discomfort was removed. She brought me my hat, and I knew I was going out into the warm sunshine. This thought, if a wordless sensation may be called a thought, made me hop and skip with pleasure. \\

\textbf{3.} 有一天,我正在玩新布娃娃时,莎莉文小姐把我的旧布娃娃也放在了我的膝盖上,然后拼写“d-o-l-l(玩偶)”,试图让我明白这个词适用于两个娃娃。那天早些时候,我们还为“mug(大杯子)”和“water(水)”这两个词争执过。莎莉文小姐努力让我明白“杯是杯,水是水”,但我总是把两者混淆。她绝望之下暂时搁置了这个话题,一有机会又重新提起。我对她反复的尝试感到不耐烦,于是抓起新娃娃摔在了地上。当我感觉到脚边破碎的娃娃碎片时,心中竟十分痛快。这次情绪爆发后,我既没有悲伤,也没有后悔。我并不爱那个娃娃。在我生活的那个寂静、黑暗的世界里,没有强烈的温柔情感。我感觉到老师把碎片扫到了壁炉边,心中涌起一股满足感——让我不快的东西终于消失了。她给我拿来帽子,我知道要去外面温暖的阳光下了。这个念头——如果这种无言的感受能被称为念头的话——让我高兴地蹦蹦跳跳。 \\

\textbf{4.} We walked down the path to the wellhouse, attracted by the fragrance of the honeysuckle with which it was covered. Some one was drawing water and my teacher placed my hand under the spout. As the cool stream gushed over one hand she spelled into the other the word water, first slowly, then rapidly. I stood still, my whole attention fixed upon the motions of her fingers. Suddenly I felt a misty consciousness as of something forgotten - a thrill of returning thought; and somehow the mystery of language was revealed to me. I knew then that ``w-a-t-e-r'' meant the wonderful cool something that was flowing over my hand. That living word awakened my soul, gave it light, hope, joy, set it free! There were barriers still, it is true, but barriers that could in time be swept away. \\

\textbf{4.} 我们沿着小路走向井房,井房上覆盖的金银花散发着芬芳,吸引着我们。有人正在打水,老师把我的手放在了水管下方。当清凉的水流涌过我的一只手时,她在我的另一只手上拼写“water(水)”,起初很慢,后来越来越快。我站在原地一动不动,全部注意力都集中在她手指的动作上。突然,我感觉到一种被遗忘的朦胧意识渐渐苏醒——一股思绪回归的悸动。不知怎的,语言的奥秘在我面前揭开了。那一刻我明白了,“w-a-t-e-r(水)”指的就是流经我手心的、那种美妙清凉的东西。这个充满生命力的词语唤醒了我的灵魂,给它带来了光明、希望、欢乐和自由!诚然,障碍依然存在,但这些障碍终将被扫除。 \\

\textbf{5.} I left the well-house eager to learn. Everything had a name, and each name gave birth to a new thought. As we returned to the house every object which I touched seemed to quiver with life. That was because I saw everything with the strange, new sight that had come to me. On entering the door I remembered the doll I had broken. I felt my way to the hearth and picked up the pieces. I tried vainly to put them together. Then my eyes filled with tears; for I realized what I had done, and for the first time I felt repentance and sorrow. \\

\textbf{5.} 我离开井房时,心中充满了学习的渴望。万物都有名字,每个名字都催生了一个新的想法。 当我们回到屋里时,我触摸到的每一件物品似乎都在因生命而颤动。 那是因为我用一种全新的、奇异的视角看待一切——这种视角刚刚降临到我身上。 一进门,我就想起了被我摔坏的娃娃。 我摸索着走到壁炉边,捡起碎片,徒劳地试图把它们拼凑起来。 这时,我的眼里充满了泪水,因为我意识到自己犯下的错,生平第一次感受到了悔恨与悲伤。 \\

\textbf{6.} I learned a great many new words that day. I do not remember what they all were; but I do know that mother, father, sister, teacher were among them - words that were to make the world blossom for me, ``like Aaron's rod, with flowers.'' It would have been difficult to find a happier child than I was as I lay in my crib at the close of the eventful day and lived over the joys it had brought me, and for the first time longed for a new day to come. \\

\textbf{6.} 那天我学会了许许多多新单词。我不记得所有单词是什么了, 但我清楚“母亲”“父亲”“姐妹”“老师”就在其中——这些词语为我打开了一个缤纷的世界,“像亚伦的权杖,绽放出鲜花”。在这个意义非凡的一天结束时,我躺在婴儿床上,重温着这一天带来的喜悦,生平第一次热切期盼着新一天的到来。此刻,再也找不到比我更幸福的孩子了。 \\

\textbf{7.} I had now the key to all language, and I was eager to learn to use it. Children who hear acquire language without any particular effort; the words that fall from others' lips they catch on the wing, as it were, delightedly, while the little deaf child must trap them by a slow and often painful process. But whatever the process, the result is wonderful. Gradually from naming an object we advance step by step until we have traversed the vast distance between our first stammered syllable and the sweep of thought in a line of Shakespeare. \\

\textbf{7.} 现在我掌握了语言的钥匙,渴望学会运用它。 听力正常的孩子学语言毫不费力,他们仿佛能轻松捕捉到别人口中说出的词语,满心欢喜。而失聪的孩子却必须通过一个缓慢且常常痛苦的过程来掌握语言。 但无论过程如何,结果都是奇妙的。渐渐地,我们从给物体命名开始,一步步前进,跨越从最初结结巴巴的音节到莎士比亚诗句中深邃思想的遥远距离。 \\

\textbf{8.} At first, when my teacher told me about a new thing I asked very few questions. My ideas were vague, and my vocabulary was inadequate; but as my knowledge of things grew, and I learned more and more words, my field of inquiry broadened, and I would return again and again to the same subject, eager for further information. Sometimes a new word revived an image that some earlier experience had engraved on my brain. \\

\textbf{8.} 起初,当老师给我讲新鲜事物时,我几乎没什么问题可问。 我的想法模糊不清,词汇量也不足。但随着我对事物的认知逐渐加深,掌握的单词越来越多,我的探索范围也扩大了。我会一次次回到同一个话题,渴望了解更多信息。 有时,一个新单词会唤醒我脑海中早已铭刻的、来自过往经历的画面。 \\

\textbf{9.} I remember the morning that I first asked the meaning of the word, ``love.'' This was before I knew many words. I had found a few early violets in the garden and brought them to my teacher. She tried to kiss me: but at that time I did not like to have any one kiss me except my mother. Miss Sullivan put her arm gently round me and spelled into my hand, ``I love Helen.'' \\

\textbf{9.} 我记得那天早晨,我第一次问起“爱”这个词的意思。那时我还不认识多少单词。 我在花园里发现了几朵早开的紫罗兰,便摘下来带给老师。 她想吻我,但那时除了母亲,我不喜欢别人吻我。 莎莉文小姐轻轻地用胳膊搂住我,在我手上拼写:“我爱海伦。” \\

\textbf{10.} ``What is love?'' I asked. \\

\textbf{10.} “什么是爱?”我问。 \\

\textbf{11.} She drew me closer to her and said, ``It is here,'' pointing to my heart, whose beats I was conscious of for the first time. Her words puzzled me very much because I did not then understand anything unless I touched it. \\

\textbf{11.} 她把我搂得更近,指着我的心说:“爱就在这里。”这是我第一次意识到心脏的跳动。 她的话让我十分困惑,因为那时我只有通过触摸才能理解事物。 \\

\textbf{12.} I smelt the violets in her hand and asked, half in words, half in signs, a question which meant, ``Is love the sweetness of flowers?'' \\

\textbf{12.} 我闻着她手中紫罗兰的香气,一半用拼写,一半用手势问道,意思是:“爱是花香吗?” \\

\textbf{13.} ``No,'' said my teacher. \\

\textbf{13.} “不是。”老师说。 \\

\textbf{14.} Again I thought. The warm sun was shining on us. \\

\textbf{14.} 我又开始思考。温暖的阳光照在我们身上。 \\

\textbf{15.} ``Is this not love?'' I asked, pointing in the direction from which the heat came. ``Is this not love?'' \\

\textbf{15.} “这难道不是爱吗?”我指着阳光传来的方向问, “这难道不是爱吗?” \\

\textbf{16.} It seemed to me that there could be nothing more beautiful than the sun, whose warmth makes all things grow. But Miss Sullivan shook her head, and I was greatly puzzled and disappointed. I thought it strange that my teacher could not show me love. \\

\textbf{16.} 在我看来,没有什么比太阳更美好的了,它的温暖让万物生长。 但莎莉文小姐摇了摇头,我陷入了深深的困惑与失望之中。 我觉得奇怪,老师为什么不能向我展示爱的样子。 \\

\textbf{17.} A day or two afterward I was stringing beads of different sizes in symmetrical groups - two large beads, three small ones, and so on. I had made many mistakes, and Miss Sullivan had pointed them out again and again with gentle patience. Finally I noticed a very obvious error in the sequence and for an instant I concentrated my attention on the lesson and tried to think how I should have arranged the beads. Miss Sullivan touched my forehead and spelled with decided emphasis, ``think.'' \\

\textbf{17.} 一两天后,我正在把不同大小的珠子按对称的组合串起来——两颗大珠子、三颗小珠子,依此类推。 我犯了很多错误,莎莉文小姐却以温柔的耐心一次次指出来。 最后,我注意到序列中一个非常明显的错误,那一刻我集中注意力在这件事上,努力思考应该如何排列珠子。 莎莉文小姐摸了摸我的额头,语气坚定地拼写出:“思考。” \\

\textbf{18.} In a flash I knew that the word was the name of the process that was going on in my head. This was my first conscious perception of an abstract idea. \\

\textbf{18.} 刹那间,我明白了这个词指的就是我脑海中正在进行的过程。 这是我第一次有意识地理解抽象概念。 \\

\textbf{19.} For a long time I was still - I was not thinking of the beads in my lap, but trying to find a meaning for ``love'' in the light of this new idea. The sun had been under a cloud all day, and there had been brief showers; but suddenly the sun broke forth in all its southern splendour. \\

\textbf{19.} 我静静地坐了很久——没有想着膝上的珠子,而是借着这个新领悟,努力寻找“爱”的含义。 那天太阳一整天都被乌云笼罩,还下了几场小雨, 但突然间,太阳冲破云层,绽放出南方特有的灿烂光芒。 \\

\textbf{20.} Again, I asked my teacher, ``Is this not love?'' \\

\textbf{20.} 我又一次问老师:“这是爱吗?” \\

\textbf{21.} ``Love is something like the clouds that were in the sky before the sun came out,'' she replied. Then in simpler words than these, which at that time I could not have understood, she explained: ``You cannot touch the clouds, you know; but you feel the rain and know how glad the flowers and the thirsty earth are to have it after a hot day. You cannot touch love either; but you feel the sweetness that it pours into everything. Without love you would not be happy or want to play.'' \\

\textbf{21.} “爱有点儿像太阳出来前天空中的云彩。”她回答道。 然后她用更简单的语言解释——那些复杂的表达我当时还无法理解。她说道:“你知道,你摸不到云彩, 但你能感受到雨水。酷热的一天过后,花儿和干渴的土地得到雨水滋润时的喜悦,你是能体会到的。 爱也是如此,你摸不到它,但你能感受到它带给万物的美好。 没有爱,你就不会快乐,也不会有玩耍的兴致。” \\

\textbf{22.} The beautiful truth burst upon my mind - I felt that there were invisible lines stretched between my spirit and the spirits of others. \\

\textbf{22.} 一个美好的真理突然涌上心头——我感觉到我的灵魂与他人的灵魂之间,延伸着无数无形的丝线。 \\

\textbf{23.} From the beginning of my education Miss Sullivan made it a practice to speak to me as she would to any hearing child; the only difference was that she spelled the sentences into my hand instead of speaking them. If I did not know the words and idioms necessary to express my thoughts she supplied them, even suggesting conversation when I was unable to keep up my end of the dialogue. \\

\textbf{23.} 从我接受教育之初,莎莉文小姐就一直像对待听力正常的孩子一样跟我交流。 唯一的不同是,她不是用嘴说,而是在我手上拼写句子。 如果我找不到合适的词汇和习语来表达想法,她会帮我补充,甚至在我无法继续对话时主动提出话题。 \\

\textbf{24.} This process was continued for several years; for the deaf child does not learn in a month, or even in two or three years, the numberless idioms and expressions used in the simplest daily intercourse. The little hearing child learns these from constant repetition and imitation. The conversation he hears in his home stimulates his mind and suggests topics and calls forth the spontaneous expression of his own thoughts. This natural exchange of ideas is denied to the deaf child.My teacher, realizing this, determined to supply the kinds of stimulus I lacked. This she did by repeating to me as far as possible, verbatim what she heard, and by showing me how I could take part in the conversation. But it was a long time before I ventured to take the initiative,and still longer before I could find something appropriate to say at the right time. \\

\textbf{24.} 这种过程持续了好多年。对于一个失聪儿童来说,我根本无法在短短一个月,乃至两三年的时间里掌握日常生活用语。 正常的孩子可以靠不断地重复和模仿来学习语言, 他们只要多倾听家里大人们谈话,最后自然就能表达出自己的思想了。 但可惜,这种交流方式对失聪儿童是行不通的。 我的莎莉文老师很清楚这一 点,她一字一句,反反复复 地教我,告诉我怎样与别人 对话。这是一个漫长的过程 过了很长时间我才有能力和 别人交流,后来又过了很长 时间,我才学会根据不同的 场合调整用词。\\

\textbf{25.} The deaf and the blind find it very difficult to acquire the amenities of conversation. How much more this difficulty must be augmented in the case of those who are both deaf and blind! They cannot distinguish the tone of the voice or, without assistance, go up and down the gamut of tones that give significance to words; nor can they watch the expression of the speaker's face, and a look is often the very soul of what one says. \\

\textbf{25.} 聋人和盲人很难领会谈话 中的细微之处。那些既聋又 盲的人遇到的困难又会大多少倍啊!他们无法辨别人们说 话的语调,没有别人的帮助, 领会不了语气的变化所包寒 的意思。他们也看不见说话 者的神色,而神色是心灵的 自然流露。 \\

\textbf{26.} It was my teacher's genius, her quick sympathy, her loving tact which made the first years of my education so beautiful. All the best of me belongs to her - there is not a talent, or an aspiration or a joy in me that has not been awakened by her loving touch. \\

\textbf{26.} 是我老师的才华, 她的同情心和她的机智, 使我第一年的学习生活 变得如此美好。我永远 也分不清,我对所有美 好事物的喜爱,有多少 是自己内心固有的,有 多少是她赐予给我的。\\

\end{CJK}
\end{document}