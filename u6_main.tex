\documentclass[12pt]{article}
\usepackage[utf8]{inputenc}
\usepackage{CJKutf8}
\usepackage{setspace}
\setstretch{1.3}
\usepackage[a4paper,margin=1in]{geometry}

\begin{document}

\begin{CJK}{UTF8}{gbsn}

\section*{UNIT 6}

What does it feel like to help dying patients through their final days? Experience it through the eyes of hospice nurse Jill Campbell, who does her job with grace, compassion, and gratitude.\\

帮助即将离世的患者度过最后的时光会是怎样的感受呢?让我们借助吉尔·坎贝尔的所见经历这一切吧。吉尔·坎贝尔把优雅、同情和感激全然融入到工作中。\\

\textbf{1.} Outside, it's noisy on this busy block of row houses in Baltimore. But inside one tidy living room, all is quiet except for the sound of a woman's raspy breathing. The patient is huddled in an easy chair under a handmade pink-and-blue afghan, a knit cap on her head and booties on her feet. She has trouble staying warm these days. Her cancer has returned with a vengeance and she has only a few weeks to life.Hospice nurse Jill Campbell kneels down beside her patient, listens to her breathing, and then checks her blood pressure. Campbell has already hauled in oxygen tanks, showed family members how to work them, organized the medicine, and assessed how her patient has been eating and sleeping.\\

\textbf{1.} 在巴尔的摩的这个由联排房屋构成的繁忙街区,外面一片喧闹,但是在里面一间洁净的卧室里,除了只能听到一位女士刺耳的呼吸声之外,周围一片寂静。这位病人蜷缩在一把安乐椅上、身上盖着一条厚厚的手工制的粉蓝色毛毯、头上戴着一顶针织帽,脚上穿着一双软毛袜。这些天来这位病人一直没办法让自己保持温暖的状态。她的癌症复发了,而且到了很严重的程度。她在世上的日子不过几个星期了。  
临终关怀护士吉尔·坎贝尔跪在她的病人身旁,听她的呼气,检查她的血压。坎贝尔已经把氧气瓶拉近了,她向病人家属展示如何使用氧气瓶,之后她又准备好药物,紧接着又评估了一下病人的饮食和睡眠状况。\\

\textbf{2.} But now is a moment to connect one-on-one. Campbell wraps her hands around the woman's hands and rubs them together to warm them. She looks into her face. ``are you feeling a little better?'' she asks softly.\\

\textbf{2.} 但是现在是坎贝尔和患者之间一对一的接触时刻。坎贝尔用自己的手捂住这位女病人的手。为了让病人的双手暖和些,她帮这位病人揉搓着双手。坎贝尔看着病人的脸,轻柔地问道:``现在感觉好一些了吗?''\\

\textbf{3.} Getting to know her patients and helping them through the toughest time of their lives is what Campbell, 43, appreciates most about being a hospice nurse. ``I don't know of another position where you can do more for people,'' she says.\\

\textbf{3.} 坎贝尔43岁,身为一名临终关怀护士,坎贝尔最为珍视的就是了解她的病人并且帮助他们度过生命中最艰难的时光。她说:``我不知道还有哪个职业能像临终关怀工作一样为人们做出更多贡献。''\\

\textbf{4.} Her patients have all been told that they have six months or less to live. Rather than continue with often-difficult or painful treatments that probably won't extend their lives, they have decided to stop trying for a cure. Instead, with the help of hospice care, they'll focus on comfort and on living whatever they have left of their lives to the fullest ---usually in their own home.\\

\textbf{4.} 她的病患都被告知他们在世上的时光最多不过6个月。与其继续一向很艰难又痛苦的治疗,而且往往这些治疗可能并不会延长他们的寿命,这些病患已经决定不再尝试寻找任何治疗手段。相反地,在临终关怀的帮助下,他们往往更注重如何舒适地生活并且尽情地享用余生。他们通常会选择在自己家里接受临终关怀。\\

\textbf{5.} Being able to die at home is a major part of the appeal of hospice, but patients and family members may not see it that way at first. ``A lot of people still view hospice as giving up and letting the disease in,'' says Campbell. That's why the decision to call in hospice care can be an incredibly difficult one for a family to make. Once they do, though, most patients and their families soon understand the value of having a team of dedicated professionals---including social workers, health aides, chaplains, and nurses---work together to provide not only physical but also emotional and spiritual support.When Campbell took the job at Gilchrist Hospice\_Care three years ago, she had the same fears as anyone about hospice. As a mother of three and a veteran nurse who'd worked in operating rooms, she expected it to be unbearable sad. But on her first home visit, she was surprised that the family members were relaxed and sharing funny stories about their dying father. ``There's still happiness in the sadness of it,'' she says.\\\\

\textbf{5.} 能够在家中逝去是人们选择临终关怀的主要原因之一。但是病人和他们的家属起初并没有认识到这一点。坎贝尔说:``很多人把临终关怀视为放弃治疗并且向疾病屈服。''这正是为什么对于家人来说做出选择临终关怀这个决定有如此艰难的原因。然而,一旦他们做了这个决定,大多数病人和家人很快就会理解其中的意义所在,即一支由社工、健康师(助理)、牧师和护士组成的专业人士通力合作为病人和家属提供身体上、感情上和精神上的支持与帮助。三年前当坎贝尔开始在吉尔克里斯特临终关怀中心工作时,她和其他人一样对临终关怀抱有相同的畏惧和忧虑。作为三个孩子的妈妈和经历丰富的手术室护士,坎贝尔原以为这份工作需要承受难以忍受的悲伤。但是当她第一次到病人家造访时,她很惊讶地看到病人的家人都很放松,并且他们能够和即将离世的父亲分享有趣的故事。坎贝尔说:``在悲伤中仍然能够感受到幸福。''\\

\textbf{6.} By spending time inside patient's homes, Campbell has witnessed the blessings of a peaceful ending to life. She's seen family members resolve longstanding, hurtful disputes and be reunited. For some patients the victories have been smaller but equally profound: a visit to a hair salon or being able to sit outside on a nice day.But getting pateients and their relatives to that place of peace and acceptance can be tough. Some families are divided or resistant to the idea of hospice. One family asked Campbell to cover her badge, thinking that if their grandmother saw the word hospice she'd give up and die.\\

\textbf{6.} 在病人家度过的时光让坎贝尔目睹了平静地向生命告别是何等蒙恩。她看到家人是如何解决那些持久又伤人的争论之后又言归于好,一团和气的。对于许多患者来讲能够去一趟美发沙龙、在晴朗的天气里去外面坐坐都是莫大的胜利。这些看似很小的事情对患者来讲却有着深远的意义。但是要想让患者和他们的家人能够平静地接受临终关怀却是一件难事。在很多家庭里成员间意见有分歧,还有些家庭对临终关怀持抵制态度。曾经有这样一个家庭,他们要求坎贝尔挡住她的徽章。因为他们担心如果祖母看到了徽章上临终关怀的字样,她将会放弃,进而离世。\\

\textbf{7.} Others, unnecessarily worried about drug addiction, won't give their sick relative pain medicine when it's needed. And some patients are afraid of taking morphine, thinking it will stop their reathing or make them feel out of it. ``Did you take the medicine?'' Campbell asks a cancer patient, who is holding her rib cage in agony. Campbell squats beside the hesitant woman and assures her she'll stay with her while she takes it, to make sure she's okay. The patient is worried she'll just sleep away the time she has left, but pain medicine often allows a person to feel better and actually do more.\\

\textbf{7.} 还有一些家庭即便在病人需要服用止痛药的时候也不让病人服用,因为他们担心病人会因服用药物而上瘾。然而这些担心都是不必要的。有些病人害怕服用吗啡,因为他们认为吗啡会让他们呼吸停止并且很不自在。``你吃药了吗?''坎贝尔问一位癌症患者,这位患者正在痛苦地抓着胸腔。坎贝尔蹲在这位犹豫的病人身旁,向她保证当她服药的时候坎贝尔会陪着她以确保她会平安无事。这位病人担心她会把所剩的时光都睡过去,但是止痛药通常能让病人感觉更好并且实际上能起到更大的作用。\\

\subsection*{Juggling Crises\\处理危机}

\textbf{8.} Many days Campbell is busy juggling crises---one patient has fallen down, another is vomiting, and another is close to daying. Other days, she delicately navigates the fears of patients and families with her gentle, grounded spirit. In home after home, she finds that people what to know the same thins: how long they have left and what the final moments will be like. Some only want to know if she can keep them calm and out of pain. She can. Others want details, so she'll explain that after they stop eating and drinking, for example, they will become semicomatose and just gradually slip away.\\

\textbf{8.} 很多天来坎贝尔一直忙于处理各种危象:一位患者跌倒了,另外一位正在呕吐、还有一位马上就要离世了。其它的日子里,坎贝尔细心地凭借自己温柔善意的、不屈不挠的精神排解病人和家属的恐惧。走访了一家又一家坎贝尔发现人们都想知道相同的事情:一是他们还有多少时光;二是最后的时刻会是什么样子。有些人只想知道坎贝尔能否让他们平静下来并且脱离痛苦。她能做到。另外一些人想知道一些细节,于是坎贝尔会向他们解释例如在他们停止进食后,他们会处于半昏迷状态并且会渐渐地离开人世。\\

\textbf{9.} Some still wonder if they could be the rare person who survives. ``Has there ever been a case where somebody walks away from this?'' one 75-year-old grandfather asks her hopefully. ``I don't know,'' Campbell says after a moment. She explains that it's hard to say with his kidney disease. ``live each day,'' she tells him. Then, noting his jokes about eating whatever he wants and having his daughter and wife wait on him, she adds with a smile, ``And obviously you are.''\\

\textbf{9.} 一些人依旧想知道他们是否会是能够幸存的少数人。曾经有一个75岁的祖父,他满怀希望地问坎贝尔``有没有哪个病人从这活着走出去?''过了一会儿,坎贝尔说:``我不知道。''她解释道对于他的这种肾病很难说能否幸存下来。她告诉老人``活好每一天。''紧接着老人开玩笑说想吃什么就吃点什么吧,并且让他的女儿和妻子等着他。注意到老人的玩笑后坎贝尔微笑着补充道:``很显然你就是这样做的,享受活着的每一天。''\\

\textbf{10.} Because many people see hospice care as the end of hope, there are even some doctors who are reluctant to bring up the option. As a result, more than a third of hospice patients don't start palliative care until they have just days left to live. Ironically, some patients who get hospice care live longer than those who don't, studies show. But many wait until it's nearly too late, and those people often sacrifice the chance for closure.\\

\textbf{10.} 因为很多人把临终关怀视作最后的希望,因此很多医生甚至都不情愿向病人提出临终关怀这项选择。而这么做的结果是多于三分之一的临终关怀患者直到生命仅剩几天时光时才开始接受临终关怀治疗。而具有讽刺意味的是:研究显示相比那些没有接受临终关怀的患者来说,接受临终关怀的患者活的时间要更长。但是很多人一直等到几乎是太晚了以至于他们经常牺牲了以临终关怀的方式向生命告别的机会。\\

\textbf{11.} One day Campbell gets a message: The man she's just seen for the first time two hours earlier has already died. ``Ooh,'' she says, letting out a long, frustrated sigh. She knows what she could have done for him if she'd had more time --- the same thing she wants for herself when her life is ending: a chance to have those last conversations, to be comfortable, at home, surrounded by loved ones.\\

\textbf{11.} 一天坎贝尔收到了一条信息:就在两小时前她第一次探望的病人辞世了。``噢,''她说,伴随着一声很长的沮丧的叹息。如果坎贝尔有更多时间的话,她知道她会为这位病人做些什么:创造机会让病人在家中舒服地被所爱的人包围,最后再和他们说说话,而这正是她所希望自己在生命尽头时所能拥有的。\\

\textbf{12.} That's why she tries to focus on what patients want. And when a terminally ill person hangs on longer than seems possible, Campbell has learned that the patient is often waiting for something to be resovled. In one case a dying woman's adult children are gathered at her bedside. One of the daughters, in particular, is heartbroken and distraught. The chaplain leads them in prayer and then the children, leaning on each other, leave the room. ``look, they're together,'' Campbell whispered to the woman, sensing she is worried about them. ``If you want, it's okay to go. They're going to be okay.'' Within minutes, the woman dies.\\

\textbf{12.} 这也正是她尽所能关注患者所需的原因所在。坎贝尔的亲身经历让她知晓当身患绝症的病人比看似可能的情况下坚持了更长的时间,这通常说明患者在等待某件事情得以解决。有这样一个案例:在一位即将离世的妇人床边聚集了她的成年子女。其中的一个女儿悲痛欲绝。牧师带领他们祷告,紧接着孩子们相互依偎走出了房间。坎贝尔意识到老妇人放心不下她的孩子们,于是对她耳语道:``看,他们在一起。如果这就是你所期望的,那就放心地走吧。他们会没事的。''几分钟后,老妇人辞世了。\\

\textbf{13.} ``People are so afraid of how it's going to end,'' Campbell says. ``But when you're been there and held their hand and watched them take their last breath, you see that it's a really poweful moment---powerful and peaceful.''\\

\textbf{13.} 坎贝尔说:``人们对如何离开人世都心存恐惧。但是一旦当你在场,握住他们的手,看着他们咽下最后一口气,你就会明白这真的是一个强大的时刻——强大又平安的时刻。''\\

\subsection*{Connecting\\联系}

\textbf{14.} Getting to know her patients is the most rewarding part of Jill Campbell's job as a hospice nurse. During the four months Campbell cares for her, Schuberth can barely speak. But by using her expressive eyes, as well as by tapping out words on an iPad, she manages to have a running conversation with Campbell about frozen yogurt, their families, and her beloved dogs.\\

\textbf{14.} 对于一位临终关怀护士来说,吉尔·坎贝尔认为她工作中最让她欣慰的就是了解她的患者们。在坎贝尔照顾舒伯茨的四个月里,舒伯茨几乎不能说话。但是舒伯茨通过在iPad上敲字,同时用她那会说话的眼睛尽量与坎贝尔进行连续的交流。她们谈论的话题包括冷冻酸奶、她们的家庭和她心爱的狗。\\

\textbf{15.} It's as if, under other circumstances, they could be friends. ``There are some patients that you become emotionally attached to,'' Campbell says. ``How could you not Linda? She's such an awesome person.'' She never leaves without giving her a big hug.\\

\textbf{15.} 如果在其它场合结实,她们很可能会成为朋友。坎贝尔说:``有一些患者会让你在感情上很依恋如果没有琳达,该怎么办呀?她是这么出色。''每次离开时坎贝尔都不忘给琳达一个大大的拥抱。\\

\textbf{16.} Every time Campbell visits Linda Schuberth, she feels as though she's getting a gift. Schuberth, 56, is suffering from amyotrophic lateral sclerosis, or Lou Gehrig's disease As an occupational therapist who specializes in swallowing disorders in children, she is losing her own ability to swallow. And the Baltimore woman, who used to compete in triathlons, is confined to a wheelchair, her limbs stiff and useless.\\

\textbf{16.} 每次坎贝尔探望琳达,坎贝尔都会感觉像是收到一份礼物一样。56岁的舒伯茨身受肌萎缩侧索硬化症,又叫卢·格里克症的困扰。作为一名职业治疗师,舒伯茨一直专治儿童吞咽混乱。然而现在她自己却丧失了吞咽能力。这位过去常常参加铁人三项竞技的巴尔的摩女士,现在却被限制在轮椅上。她的四肢坚硬、派不上用场了。\\

\textbf{17.} Campbell is struck by the irony of that. Yet Schuberth perseveres with yoga, pool therapy, and a specially equipped stationary bicycle. ``What she's doing is so amazing,'' Campbell says. ``She's inspiring.''\\

\textbf{17.} 这样具有讽刺意味的事让坎贝尔倍受打击。然而舒伯茨却不屈不挠,她坚持不懈地练习瑜伽,坚持水池疗法,并且在专门配备的固定自行车上锻炼。坎贝尔说:``她说做的一切真是了不起。她总是让人振奋,鼓舞人心。''\\

\textbf{18.} On a day in late December, Schuberth's eyes are bright and despite the fact that her facial muscles are getting weaker, she pulls off a big smile. ``Did you see your friends today?'' Campbell asks, referring to the staff at Kennedy Krieger Institution in Baltimore, where Schuberth worked as an occupational therapist and now gets her own therapy. Schuberth nods and leans forward to use her iPad. Her fingers are curled and rigid, so a splint helps to separate one finger for typing. For several long moments she taps on the keyboard. Then a computerized voice speaks out what Schuberth wants to say about her colleagues: ``Miss so much.''\\

\textbf{18.} 在12月末的一天,舒伯茨眼睛明亮,尽管她的面部肌肉越来越无力,但她依旧努力呈现一个灿烂的微笑。坎贝尔问道:``今天看到你的朋友们了吗?''这些朋友就是巴尔的摩肯尼迪·克里格研究所的工作人员,她们都是舒伯茨的同事。过去,舒伯茨在这里从事专业治疗师的工作,而现在她自己在这里接受治疗。舒伯茨点点头,身体向前倾斜来使用她的iPad。她的手指蜷曲在一起,很坚硬,因此需要借助一个夹板把一个手指头分出来用来敲字。舒伯茨在键盘上敲了很长时间,随即一个电脑化的声音大声说出了舒伯茨关于她的同事们所要表达的内容:``非常想念。''\\

\textbf{19.} ``She thinks her ears stick out with the new haircut, but I think she looks really good,'' says Schuberth's husband of 21 years, Ken Schuberth, MD, a pediatric allergis in Baltimore. He tells Campbell that he never could keep up with his athletic wife, who was passionate about running, cycling, and yoga. Campbell savors seeing them together.\\

\textbf{19.} ``她认为刚刚剪过头之后她的耳朵显得太突出了,但是我认为她看上去真的很好,''舒伯茨的丈夫说。舒伯茨的丈夫,肯·舒伯茨是一位医学博士,在巴尔的摩工作,是一名小儿过敏症专科医师。他们结婚已经有21年了。他告诉坎贝尔他从来都跟不上他这位像运动员一样的妻子,她狂热于跑步、骑单车和瑜伽。看到他们在一起让坎贝尔很欣慰。\\

\textbf{20.} For months ALS has made it harder for Schuberth to swallow anything, including the pills that ease the pain and stiffness in her limbs. One day in late February, she realizes that she can no longer swallow at all. She painstakingly types a message to Campbell and Ken: ``I'm so tired....'' She doesn't want to even consider using a feeding tube. They know what that means, and a tear runs down Ken's face. ``It's breaking my heart,'' Campbell says.\\

\textbf{20.} 几个月来肌萎缩侧索硬化症让舒伯茨吞咽东西变得更加困难,其中包括能减轻痛苦和缓解四肢僵硬的药丸。在二月末的一天,她意识到她再也不能吞咽了。她痛苦地给肯和坎贝尔敲打出一个信息:``我太累了……''她甚至不考虑使用饲管。他们知道这意味着什么,一滴眼泪流淌在肯的脸上。坎贝尔说:``这让我的心都碎了。''\\

\textbf{21.} On a dark late-winter day, Schuberth is bundled under a blue fleece blanket in a hospital bed in her comfortable living room. Friends come in twos and threes to her bedside. ``We're reliving so many wonderful memories,'' Ken says. ``Although she can't speak, she really hears everything that people say.'' Just that morning, he laughs, his wife had signaled him to take out the recycling.But now, as the pain and sorrow overwhelm her, Schuberth begins to sob. Ken strokes his wife's hair as Campbell soothes her with medicine, words, and touches. It's starting to rain, and it's time for Campbell to get home to her three kids. She whispers good-bye and gives Schuberth one last, gentle kiss on the forehead. The groundwork is laid for the peaceful ending Schuberth wants. Later, with her family gathered around her, she quietly passes away.\\

\textbf{21.} 在深冬的一个阴天,舒伯茨躺在她舒适的卧室里的医院用病床上,身上包裹着一条蓝色的羊毛毯子。朋友们三三两两来到她的病床边。肯说:``我们一起回忆了那么多美好时光。尽管舒伯茨不能说话,但是她真的能够听到大家说的每件事。''就在那天下午,肯笑了,因为舒伯茨示意他把回收的东西取出。但是现在,当她被痛苦和悲伤彻底压倒时,舒伯茨开始抽泣。当坎贝尔用药物、语言和抚摸来减轻舒伯茨的痛苦时,肯轻抚着妻子的头发。开始下雨了,这正是坎贝尔要赶回家的时候,家里还有三个孩子在等着她。她轻声对舒伯茨说再见,同时温柔地亲吻了一下舒伯茨的额头,这是她给舒伯茨的最后一吻。所有一切都准备好了,为了让舒伯茨能够如愿地安详离世。随后,舒伯茨的家人围聚在她的身旁,她平静地走了。\\

\textbf{22.} If you knew you were going to die, what would you want?\\

\textbf{22.} 如果你知道自己来日不多的话,你会如何选择?\\

\subsection*{英译汉}

\textbf{1)} Being able to die at home is a major part of the appeal of hospice, but patients and family members may not see it that way at first. ``A lot of people still view hospice as giving up and letting the disease win,'' says Campbell. That's why the decision to call in hospice care can be an incredibly difficult one for a family to make. Once they do, though, most patients and their families soon understand the value of having a team of dedicated professionals -- including social workers, health aides, chaplains, and nurses -- work together to provide not only physical but also emotional and spiritual support.\\

能够在家中逝去是人们选择临终关怀的主要原因之一。但是病人和他们的家属起初并没有认识到这一点。坎贝尔说:``很多人把临终关怀视为放弃治疗并且向疾病屈服。''这正是为什么对于家人来说做出选择临终关怀这个决定有如此艰难的原因。然而,一旦他们做了这个决定,大多数病人和家人很快就会理解其中的意义所在,即一支由社工、健康师(助理)、牧师和护士组成的专业人士通力合作为病人和家属提供身体上、感情上和精神上的支持与帮助。\\

\subsection*{汉译英}

\textbf{1)} 美国的第一家临终关怀院成立于1974年。此后,临终关怀的理念迅速流行。如今,半数美国人在关怀院度过人生最后时光。美国 $75\%$ 的绝症患者选择放弃治疗,听凭天命。对于如何关心临终病人,欧洲人的态度也大有改观,不再回避这个话题。但临终关怀运动仍面临挑战,使得目前用于缓痛护理上的资源投入大量流失,即便是在发达国家,此种情况也有发生。临终关怀的对象多数是癌症患者。癌症恶化到一定阶段后,患者一般都去日无多。但目前对病人的救治,要么把他们看作不死,要么把他们看作必死。在发达国家,存在着大量健康状况不佳,但并未患绝症的老年人口,这一人群在发展中国家也将逐渐扩大。这一问题只有通过医院、养老院和家庭三方的共同努力才能解决。照顾老年人并不简单;那些身患多种疾病的老年人经常在家庭医生、健康顾问或是专科医生之间无所适从。对此,依据本国医疗及政治环境的不同,各国有着不同的解决途径。有关实验步骤和措施在道德上的可接受性,应该在一开始就要在研究者和被试之间达成明确和公正的一致意见,明确每个人的责任。研究者有责任遵守和履行协议中规定的承诺和义务。\\

America's first hospice was founded in 1974, and the idea spread rapidly. Half of all Americans will now use hospice care at some point in their lives, and around $75\%$ of deaths in American hospitals occur after an explicit decision not to intervene. In Europe, too, there has been a revolution in attitudes to care for people who are nearing the end of life, and in people's willingness to broach the subject. Yet for all its successes, the hospice movement faces challenges that will far outstrip the resources now dedicated to palliative care, even in the richest countries.Hospices are generally associated with cancer, where after a certain stage life expectancy is short and fairly predictable. But treating people as ``either temporarily immortal or dying'' is the current habit.An important category of people, already huge in the rich world and soon to grow in developing countries, consists of elderly people who will never be well, but have no idea when they will die. There is no single answer: hospitals, nursing homes and family care will all play a role. Looking after the old is bound to be complicated; elderly people with several diseases can all too easily find themselves bounced from family doctor to health adviser to specialists in one field after another. But depending on their medical and political culture, different countries are tackling the problem in different ways.As regards the moral acceptability of experimental procedures and measures, there must be a clear and fair agreement reached by researchers and subjects at the outset in order to clarify each other's responsibility. Researchers have a responsibility to abide by and comply with the provisions of commitments and obligations in the agreement.\\

\end{CJK}
\end{document}